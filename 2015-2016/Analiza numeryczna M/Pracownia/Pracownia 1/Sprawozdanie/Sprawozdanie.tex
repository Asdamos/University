\documentclass{article}

\usepackage{polski}
\usepackage[utf8]{inputenc}

\usepackage{fancyhdr} % Required for custom headers
\usepackage{lastpage} % Required to determine the last page for the footer
\usepackage{extramarks} % Required for headers and footers
\usepackage[usenames,dvipsnames]{color} % Required for custom colors
\usepackage{graphicx} % Required to insert images
\usepackage{listings} % Required for insertion of code
\usepackage{courier} % Required for the courier font
\usepackage{lipsum} % Used for inserting dummy 'Lorem ipsum' text into the template\
\usepackage{amsfonts}
\usepackage{amsthm}
\usepackage{hyperref}
\usepackage{tikz}
\usepackage{amsmath}
\usepackage{pdfpages}
\usepackage{mathtools}
\usepackage{amsthm}


\DeclareUnicodeCharacter{00A0}{ }


\newtheorem{thm}{Twierdzenie}
\newtheorem{remark}{Uwaga}
\newtheorem{lemat}{Lemat}
\newtheorem{wniosek}{Wniosek}
\newtheorem{definicja}{Definicja}
\newtheorem{ciekawostka}{Ciekawostka}
\newtheorem{przyklad}{Przykład}


\newenvironment{prooff}{\paragraph{Dowód:}}{\hfill$\square$}
\newenvironment{rozw}{\paragraph{Rozwiązanie:}}{\hfill}

% Margins
\topmargin=-0.45in
\evensidemargin=0in
\oddsidemargin=0in
\textwidth=6.5in
\textheight=9.0in
\headsep=0.25in

\linespread{1.1} % Line spacing

% Set up the header and footer
\pagestyle{fancy}
\lhead{\hmwkAuthorName} % Top left header
\rhead{\firstxmark} % Top right header
\lfoot{\lastxmark} % Bottom left footer
\cfoot{} % Bottom center footer
\rfoot{Page\ \thepage\ of\ \protect\pageref{LastPage}} % Bottom right footer
\renewcommand\headrulewidth{0.4pt} % Size of the header rule
\renewcommand\footrulewidth{0.4pt} % Size of the footer rule

\setlength\parindent{0pt} % Removes all indentation from paragraphs
%----------------------------------------------------------------------------------------
%	DOCUMENT STRUCTURE COMMANDS
%	Skip this unless you know what you're doing
%----------------------------------------------------------------------------------------

% Header and footer for when a page split occurs within a problem environment
\newcommand{\enterProblemHeader}[1]{
\nobreak\extramarks{#1}{#1 continued on next page\ldots}\nobreak
\nobreak\extramarks{#1 (continued)}{#1 continued on next page\ldots}\nobreak
}

% Header and footer for when a page split occurs between problem environments
\newcommand{\exitProblemHeader}[1]{
\nobreak\extramarks{#1 (continued)}{#1 continued on next page\ldots}\nobreak
\nobreak\extramarks{#1}{}\nobreak
}

\setcounter{secnumdepth}{0} % Removes default section numbers
\newcounter{homeworkProblemCounter} % Creates a counter to keep track of the number of problems

\newcommand{\homeworkProblemName}{}
\newenvironment{homeworkProblem}[1][Zadanie \arabic{homeworkProblemCounter}]{ % Makes a new environment called homeworkProblem which takes 1 argument (custom name) but the default is "Problem #"
\stepcounter{homeworkProblemCounter} % Increase counter for number of problems
\renewcommand{\homeworkProblemName}{#1} % Assign \homeworkProblemName the name of the problem
\section{\homeworkProblemName} % Make a section in the document with the custom problem count
\enterProblemHeader{\homeworkProblemName} % Header and footer within the environment
}{
\exitProblemHeader{\homeworkProblemName} % Header and footer after the environment
}

\newcommand{\problemAnswer}[1]{ % Defines the problem answer command with the content as the only argument
\noindent\framebox[\columnwidth][c]{\begin{minipage}{0.98\columnwidth}#1\end{minipage}} % Makes the box around the problem answer and puts the content inside
}

\newcommand{\homeworkSectionName}{}
\newenvironment{homeworkSection}[1]{ % New environment for sections within homework problems, takes 1 argument - the name of the section
\renewcommand{\homeworkSectionName}{#1} % Assign \homeworkSectionName to the name of the section from the environment argument
\subsection{\homeworkSectionName} % Make a subsection with the custom name of the subsection
\enterProblemHeader{\homeworkProblemName\ [\homeworkSectionName]} % Header and footer within the environment
}{
\enterProblemHeader{\homeworkProblemName} % Header and footer after the environment
}

\usepackage{listings} % Required for inserting code snippets
\usepackage[usenames,dvipsnames]{color} % Required for specifying custom colors and referring to colors by name

\definecolor{DarkGreen}{rgb}{0.0,0.4,0.0} % Comment color
\definecolor{highlight}{RGB}{255,251,204} % Code highlight color

\lstdefinestyle{Style1}{ % Define a style for your code snippet, multiple definitions can be made if, for example, you wish to insert multiple code snippets using different programming languages into one document
language=Perl, % Detects keywords, comments, strings, functions, etc for the language specified
backgroundcolor=\color{highlight}, % Set the background color for the snippet - useful for highlighting
basicstyle=\footnotesize\ttfamily, % The default font size and style of the code
breakatwhitespace=false, % If true, only allows line breaks at white space
breaklines=true, % Automatic line breaking (prevents code from protruding outside the box)
captionpos=b, % Sets the caption position: b for bottom; t for top
commentstyle=\usefont{T1}{pcr}{m}{sl}\color{DarkGreen}, % Style of comments within the code - dark green courier font
deletekeywords={}, % If you want to delete any keywords from the current language separate them by commas
%escapeinside={\%}, % This allows you to escape to LaTeX using the character in the bracket
firstnumber=1, % Line numbers begin at line 1
frame=single, % Frame around the code box, value can be: none, leftline, topline, bottomline, lines, single, shadowbox
frameround=tttt, % Rounds the corners of the frame for the top left, top right, bottom left and bottom right positions
keywordstyle=\color{Blue}\bf, % Functions are bold and blue
morekeywords={}, % Add any functions no included by default here separated by commas
numbers=left, % Location of line numbers, can take the values of: none, left, right
numbersep=10pt, % Distance of line numbers from the code box
numberstyle=\tiny\color{Gray}, % Style used for line numbers
rulecolor=\color{black}, % Frame border color
showstringspaces=false, % Don't put marks in string spaces
showtabs=false, % Display tabs in the code as lines
stepnumber=5, % The step distance between line numbers, i.e. how often will lines be numbered
stringstyle=\color{Purple}, % Strings are purple
tabsize=2, % Number of spaces per tab in the code
}

% Create a command to cleanly insert a snippet with the style above anywhere in the document
\newcommand{\insertcode}[2]{\begin{itemize}\item[]\lstinputlisting[caption=#2,label=#1,style=Style1]{#1}\end{itemize}} % The first argument is the script location/filename and the second is a caption for the listing

%----------------------------------------------------------------------------------------
%	NAME AND CLASS SECTION
%----------------------------------------------------------------------------------------

\newcommand{\hmwkTitle}{Pracownia 1} % Assignment title
\newcommand{\hmwkDueDate}{} % Due date
\newcommand{\hmwkClass}{Analiza numeryczna} % Course/class
\newcommand{\hmwkClassTime}{} % Class/lecture time
\newcommand{\hmwkClassInstructor}{} % Teacher/lecturer
\newcommand{\hmwkAuthorName}{Bartosz Bednarczyk, Jan Góra} % Your name

%----------------------------------------------------------------------------------------
%	TITLE PAGE
%----------------------------------------------------------------------------------------

\title{
\vspace{2in}
\textmd{\textbf{\hmwkClass:\ \hmwkTitle}}\\
\normalsize\vspace{0.1in}\small{Due\ on\ \hmwkDueDate}\\
\vspace{0.1in}\large{\textit{\hmwkClassInstructor\ \hmwkClassTime}}
\vspace{3in}
}

\author{\textbf{\hmwkAuthorName}}
\date{} % Insert date here if you want it to appear below your name

%----------------------------------------------------------------------------------------

\begin{document}

\tableofcontents

\section{Przegląd podstawowych zagadnień związanych z wielomianami}

\subsection{Postawowe definicje}

\begin{definicja}
Wielomianem stopnia $n \in \mathbb{N}$ nad ciałem $\mathbb{K}$ będziemy nazywać przekształcenie $\mathbb{K}^n \mapsto \mathbb{K}$ zadane wzorem $W(x) = a_0 + a_1x + \ldots + a_n x^n$, gdzie $a_i$ to pewne  współczynniki z ciała $\mathbb{K}$.
\end{definicja}

\begin{definicja}
Niech $W$ będzie pewnym wielomianem (nad ciałem $\mathbb{K}$). Liczbę $a$, taką że $W(a) = 0$, będziemy nazywać pierwiastkiem wielomianu.
\end{definicja}

\begin{remark}
Z faktu, że wielomian $W$ ma współczynniki z ciała $\mathbb{K}$, nie wynika fakt, że jego pierwiastki również będą należeć do $\mathbb{K}$. Klasycznym przykładem jest wielomian $x^2 + 1$, który ma współczynniki rzeczywiste, a jego pierwiastkami są liczby zespolone.
\end{remark}

\begin{remark}
Istnieją takie ciała $\mathbb{K}$, że dla dowolnego wielomianu stopnia większego od $0$ wszystkie jego pierwiastki należą do $\mathbb{K}$. Ciała takie będziemy nazywać algebraicznie domkniętymi. Przykładem takiego ciała jest $\mathbb{C}$, czego nie będziemy dowodzić. 
\end{remark}

Podczas całego tego sprawozdania będziemy zajmować się następującym problemem:

\begin{center}
\begin{tikzpicture}
\node [draw={black}, fill=black!10, very thick, rectangle, rounded corners, inner sep=12pt, inner ysep=12pt] (box){%
    \begin{minipage}{.9\textwidth}

	Niech $W$ będzie wielomianem. Celem jest znaleźć zbiór $ker(W) = \lbrace a \ | \ W(a) = 0 \rbrace $.
    \end{minipage}
};
\node[fill={black}, text=white, rounded corners, right=10pt] at (box.north west) {Problem znajdowania miejsc zerowych wielomianu};
\end{tikzpicture}
\end{center}

Powyższy problem, choć pozornie prosty, jest sformułowany bardzo ogólnie. Na potrzeby tej pracy od tej pory ograniczymy się tylko do $\mathbb{R}$ oraz $\mathbb{C}$, choć nic nie staje na przeszkodzie by poeksperymentować z innymi ciałami. Aktualnie nie wiemy czy każdy wielomian ma pierwiastki, a jeśli ma to czy ich zbiór jest skończony. Nie znamy również żadnych metod rozwiązywania $W(x) = 0$. By lepiej zrozumieć podane zagadnienie przejdźmy przez ciąg różnych definicji, algorytmów, twierdzeń i lematów związanych z wielomianami (warto je zrozumieć, gdyż kolejne rozdziały będą z nich korzystać).

\begin{thm}
Każdy wielomian $W(x)$ nad $\mathbb{C}$ stopnia $n \in \mathbb{N}_+$ ma co najmniej jeden pierwiastek.
\end{thm}

\begin{proof}
To twierdzenie jest nazywane zasadniczym twierdzeniem algebry. Dowód \cite{leja} s. 105.
\end{proof}

\begin{wniosek}
$| ker(W) | \leq n$, gdzie $n$ to stopień wielomianu $W$.	
\end{wniosek}


\subsection{Postać iloczynowa wielomianu i dzielenie wielomianu}


\begin{definicja}
Wielomian $W(x)$ nazywamy podzielnym przez wielomian $P(x)$, różny od wielomianu zerowego, wtedy i tylko wtedy, gdy istnieje taki wielomian $Q(x)$, że $W(x) = Q(x) * P(x)$. Wielomian $Q(x)$ nazywamy ilorazem wielomianu $W(x)$ przez $P(x)$. Mówimy, że wielomian $P(x)$ jest dzielnikiem wielomianu $W(x)$.
\end{definicja}


\begin{definicja}
Dowolny wielomian $W(x)$ możemy zapisać jako $W(x) = P(x) \cdot Q(x) + R(x)$ dla pewnych wielomianów $P, Q, R$. Mówimy, że wielomian $W(x)$ jest podzielny przez $Q(x)$ jeżeli $R(x) = 0$. 
\end{definicja}

\begin{thm}
Wielomian $W(x)$ jest podzielny przez wielomian $Q(x) = (x-a)$ wtedy i tylko wtedy, gdy $W(a) = 0$.	
\end{thm}

\begin{proof}
W \cite{kostrikin}	
\end{proof}

Chcielibyśmy umieć w efektywny sposób realizować procedurę dzielenie wielomianu przez jednomiany postaci $x-a$. Służy do tego następujący algorytm:

\begin{enumerate}
\item $P(x) = a_0 + a_1 x + a_2 x^2 + \ldots + a_n x^n$
\item Niech $\alpha = a_n$
\item Kolejno dla $k = n-1, n-2, \ldots 0$ wykonaj $\alpha := a_k + x \alpha$.
\item Wynik to $p(x) = \alpha$.
\end{enumerate}

Dokładny opis metody oraz jej analizę możemy znaleźć w \cite{kincaid} (s 103).\\

\subsection{Pochodna wielomianu i jej obliczanie}

TUTAJ DODAĆ OPIS DOTYCZĄCY POCHODNYCH I OBLICZANIA ICH DLA WIELOMIANU.

\section{Metoda Newtona oraz wielowymiarowa metoda Newtona}

\subsection{Opis klasycznej metody Newtona}

Klasyczną metodą Newtona zastosowaną dla pewnego punktu startowego $p$ oraz funkcji $f : \mathbb{R} \mapsto \mathbb{R}$ klasy $C^{1}$ nazywać będziemy metodę iteracyjną postaci:

$$
x_n = \left\{\begin{matrix}
p, & gdy \ n = 0\\ 
x_{n-1} - \frac{f(x_{n-1})}{f'(x_{n-1})}, & w.p.p. 
\end{matrix}\right.
$$

Analizę klasycznej metody Newtona można znaleźć w \cite{kincaid} na stronach 71-81.

\subsection{Zastosowanie klasycznej metody Newtona do szukania zer wielomianu}

Jeśli mamy wielomian o współczynnikach i pierwiastkach rzeczywistych możemy policzyć jego pierwiastki za pomocą klasycznej metody Newtona. Podstawiamy za $f$ z poprzedniego opisu nasz wielomian, a $f'$ to jego pochodna. Po znalezieniu jednego pierwiastka (nazwijmy go $a$) dzielimy nasz wielomian przez $x-a$ i uruchamiamy program dla otrzymanego wielomianu. Proces kontynuujemy tak długo, aż dojdziemy do wielomianu o stopniu $0$.

\begin{remark}
Wartość w punkcie wielomianu i jego pochodnej możemy wyznaczyć z pomocą schematu Hornera, który był omówiony wcześniej (w kodzie przykładowym skorzystaliśmy z funkcji bibliotecznych dla większej czytelności).
\end{remark}


\lstset{language=Haskell, label=DescriptiveLabel, frame=shadowbox}
\lstinputlisting{klasyczna_metoda_newtona.jl}

\begin{remark}
Powyższa metoda nie nadaje się do obliczania miejsc zerowych wielomianu, którego pierwiastki są zespolone (z powodu tego, że operujemy tutaj na tylko rzeczywistych przybliżeniach $x_n$).
\end{remark}

\subsection{Metoda Newtona dla wielu funkcji wielu zmiennych}

Załóżmy, że mamy do rozwiązania układ równań:

$$\left\{\begin{matrix}
f_1(x_1, x_2, \ldots, x_n) = 0 \\ 
f_2(x_1, x_2, \ldots, x_n) = 0 \\ 
\ldots\\ 
f_n(x_1, x_2, \ldots, x_n) = 0
\end{matrix}\right.,$$

gdzie $f_i \in \mathbb{R}^n \mapsto \mathbb{R}^n$ jest klasy $C^{1}$.

Każdą z tych funkcji możemy rozpisać ze wzoru Taylora jako:

$$0 = f_i(x_1 + h_1, x_2 + h_2, \ldots, x_n + h_n) \approx f_i(x_1, x_2, \ldots, x_n) + \sum_{j=1}^n h_j \cdot \frac{\partial f_i}{\partial x_j} (x_1, x_2, \ldots x_n)$$

Powyższy układ możemy zapisać w postaci macierzowej:

$$
\begin{pmatrix}
f_1(x_1, x_2, \ldots, x_n)\\ 
f_2(x_1, x_2, \ldots, x_n)\\ 
\ldots\\ 
f_n(x_1, x_2, \ldots, x_n)\\ 

\end{pmatrix} = \begin{pmatrix}
\frac{\partial f_1}{\partial x_1}(x_1, x_2, \ldots x_n) & \frac{\partial f_1}{\partial x_2}(x_1, x_2, \ldots x_n) &  \ldots &  & \frac{\partial f_1}{\partial x_1}(x_1, x_2, \ldots x_n) \\ 
\frac{\partial f_2}{\partial x_1}(x_1, x_2, \ldots x_n) & \frac{\partial f_2}{\partial x_2}(x_1, x_2, \ldots x_n) &  \ldots &  & \frac{\partial f_2}{\partial x_1}(x_1, x_2, \ldots x_n) \\ 
 \ldots & \ldots &  \ \ldots & \\ 
 \ldots & \ldots &  \ \ldots & \\ 
\frac{\partial f_n}{\partial x_1}(x_1, x_2, \ldots x_n) & \frac{\partial f_n}{\partial x_2}(x_1, x_2, \ldots x_n) &  \ldots &  & \frac{\partial f_n}{\partial x_1}(x_1, x_2, \ldots x_n) \\ 
\end{pmatrix} \begin{pmatrix}
h_1\\ 
h_2\\ 
\ldots \\ 
\ldots \\ 
h_n\\ 

\end{pmatrix}
$$

Aby nieco skrócić ten układ, będziemy go zapisywać jako $F(X) = -J \cdot H$.
Jeśli macierz $J$ jest nieosobliwa, to układ ma rozwiązanie w postaci:
$$ -J^{-1} \cdot F(X) = H$$

Ostatecznie wzór Newtona dla układu funkcji wielu zmiennych możemy wzorem:

$$ X_{k+1} = X_k + H_k = X_k - J^{-1}(X_k) F(X_k)$$


\subsection{Metoda Newtona dla funkcji zespolonych}

\begin{lemat}
Dowolną funkcję analityczną $f : \mathbb{C} \mapsto \mathbb{C}$ możemy zapisać jako $$f(z) = f(x+yi) = P(x,y) + i Q(x,y),$$ gdzie $x,y \in \mathbb{R}$, $P(x,y) \in \mathbb{R}, Q(x,y) \in \mathbb{R}$	
\end{lemat}

\begin{przyklad}

$$f(z) = z^3 - 2z = f(x+iy) = (x+iy)^3 - 2(x+iy) = (x^3 - 3xy^2 - 2x) + i(3x^2y - y^3 - 2y) = P(x,y) + iQ(x,y)$$

\end{przyklad}


Niech $f(z) = P(x,y) + iQ(x,y)$. Równanie $f(z) = 0$ możemy sprowadzić do układu równań $Q(x,y) = 0$ i $P(x,y) = 0$. Taki układ równań rozwiązujemy za pomocą metody Newtona dla funkcji wielu zmiennych.

$$v_{n+1} = v_n - \frac{f(v_n)}{f'(v_n)}$$

$$
\begin{pmatrix} x_{n+1}\\ y_{n+1}\\ \end{pmatrix} =  \begin{pmatrix} x_n\\ y_n\\ \end{pmatrix} - J^{-1} \begin{pmatrix} P(x_n, y_n)\\ Q(x_n, y_n)\\ \end{pmatrix} = 
\begin{pmatrix} x_n\\ y_n\\ \end{pmatrix} - \begin{pmatrix}
\frac{\partial P}{\partial x}(x_n, y_n) & \frac{\partial P}{\partial y} (x_n, y_n)\\ 
\frac{\partial Q}{\partial x}(x_n, y_n) & \frac{\partial Q}{\partial y}(x_n, y_n)  
\end{pmatrix}^{-1} \begin{pmatrix} P(x_n, y_n)\\ Q(x_n, y_n)\\ \end{pmatrix} $$


Ponieważ wielomian jest funkcją holomorficzną, to zachodzi równanie Cauchy'ego-Riemanna:

$$\frac{\partial P}{\partial x} = \frac{\partial Q}{\partial y}, \ - \frac{\partial P}{\partial y} =\frac{\partial Q}{\partial x}$$

Oznaczając $P = P(x_n, y_n), Q = Q(x_n, y_n), P_x = \frac{\partial P}{\partial x}(x_n, y_n)$, $Q_x = \frac{\partial Q}{\partial x}(x_n, y_n)$ oraz korzystając ze wzoru na macierz odwrotną możemy uprościć wzór na metodę Newtona do postaci:

$$
\begin{pmatrix} x_{n+1}\\ y_{n+1}\\ \end{pmatrix} = \begin{pmatrix} x_n - \frac{ P P_x + Q Q_x}{Px^2 + Qx^2}\\ y_n - \frac{P P_y + Q Q_y}{Px^2 + Qx^2} \\ \end{pmatrix}
$$

\begin{thebibliography}{99}
\bibitem{leja} Leja, Franciszek.
\emph{Funkcje zespolone},
Warszawa : PWN, 1976

\bibitem{kostrikin} Aleksiej I., Kostrikin
\emph{Wstęp do algebry. Podstawy algebry},
Warszawa : PWN, 2008

\bibitem{kincaid} David Kincaid, Ward Cheney
\emph{Analiza numeryczna},
Warszawa : WNT, 2006

\bibitem{complexnewton} 
Lily Yau, Adi Ben-Israel
\emph{The Newton and Halley Methods for Complex Roots},
The American Mathematical Monthly 105(1998), 806–818


\end{thebibliography}

\end{document}