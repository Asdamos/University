% Bartosz Bednarczyk, Jan Góra
% Analiza numeryczna sprawozdanie

\documentclass{article}

\usepackage{polski}
\usepackage[utf8]{inputenc}

\usepackage{fancyhdr} % Required for custom headers
\usepackage{lastpage} % Required to determine the last page for the footer
\usepackage{extramarks} % Required for headers and footers
\usepackage[usenames,dvipsnames]{color} % Required for custom colors
\usepackage{graphicx} % Required to insert images
\usepackage{listings} % Required for insertion of code
\usepackage{courier} % Required for the courier font
\usepackage{lipsum} 
\usepackage{amsfonts}
\usepackage{amsthm}
\usepackage{hyperref}
\usepackage{tikz}
\usepackage{amsmath}
\usepackage{pdfpages}
\usepackage{mathtools}
\usepackage{amsthm}

\DeclareUnicodeCharacter{00A0}{ }

\newtheorem{thm}{Twierdzenie}
\newtheorem{remark}{Uwaga}
\newtheorem{lemat}{Lemat}
\newtheorem{wniosek}{Wniosek}
\newtheorem{definicja}{Definicja}
\newtheorem{ciekawostka}{Ciekawostka}
\newtheorem{przyklad}{Przykład}


\newenvironment{prooff}{\paragraph{Dowód:}}{\hfill$\square$}
\newenvironment{rozw}{\paragraph{Rozwiązanie:}}{\hfill}


\usepackage{inconsolata} % very nice fixed-width font included with texlive-full
\usepackage[usenames,dvipsnames]{color} % more flexible names for syntax highlighting colors
\usepackage{listings}

\lstset{
basicstyle=\small\ttfamily, 
columns=fullflexible, % make sure to use fixed-width font, CM typewriter is NOT fixed width
numbers=left, 
numberstyle=\small\ttfamily\color{Gray},
stepnumber=1,              
numbersep=10pt, 
numberfirstline=true, 
numberblanklines=true, 
tabsize=4,
lineskip=-1.5pt,
extendedchars=true,
breaklines=true,        
keywordstyle=\color{Blue}\bfseries,
identifierstyle=, % using emph or index keywords
commentstyle=\sffamily\color{OliveGreen},
stringstyle=\color{Maroon},
showstringspaces=false,
showtabs=false,
upquote=false,
texcl=true, % interpet comments as LaTeX
    literate={á}{{\'a}}1 {ã}{{\~a}}1 {é}{{<}}1,
inputencoding=utf8
}

\lstdefinelanguage{julia}
{
  keywordsprefix=\@,
  morekeywords={
    exit,whos,edit,load,is,isa,isequal,typeof,tuple,ntuple,uid,hash,finalizer,convert,promote,
    subtype,typemin,typemax,realmin,realmax,sizeof,eps,promote_type,method_exists,applicable,
    invoke,dlopen,dlsym,system,error,throw,assert,new,Inf,Nan,pi,im,begin,while,for,in,return,
    break,continue,macro,quote,let,if,elseif,else,try,catch,end,bitstype,ccall,do,using,module,
    import,export,importall,baremodule,immutable,local,global,const,Bool,Int,Int8,Int16,Int32,
    Int64,Uint,Uint8,Uint16,Uint32,Uint64,Float32,Float64,Complex64,Complex128,Any,Nothing,None,
    function,type,typealias,abstract
  },
  sensitive=true,
  morecomment=[l]{\#},
  morestring=[b]',
  morestring=[b]" 
}

% Margins
\topmargin=-0.45in
\evensidemargin=0in
\oddsidemargin=0in
\textwidth=6.5in
\textheight=9.0in
\headsep=0.25in

\linespread{1.1} % Line spacing

% Set up the header and footer
\pagestyle{fancy}
\lhead{\hmwkAuthorName} % Top left header
\rhead{\firstxmark} % Top right header
\lfoot{\lastxmark} % Bottom left footer
\cfoot{} % Bottom center footer
\rfoot{Page\ \thepage\ of\ \protect\pageref{LastPage}} % Bottom right footer
\renewcommand\headrulewidth{0.4pt} % Size of the header rule
\renewcommand\footrulewidth{0.4pt} % Size of the footer rule

\setlength\parindent{0pt} % Removes all indentation from paragraphs
%----------------------------------------------------------------------------------------
%	DOCUMENT STRUCTURE COMMANDS
%	Skip this unless you know what you're doing
%----------------------------------------------------------------------------------------

% Header and footer for when a page split occurs within a problem environment
\newcommand{\enterProblemHeader}[1]{
\nobreak\extramarks{#1}{#1 continued on next page\ldots}\nobreak
\nobreak\extramarks{#1 (continued)}{#1 continued on next page\ldots}\nobreak
}

% Header and footer for when a page split occurs between problem environments
\newcommand{\exitProblemHeader}[1]{
\nobreak\extramarks{#1 (continued)}{#1 continued on next page\ldots}\nobreak
\nobreak\extramarks{#1}{}\nobreak
}

\setcounter{secnumdepth}{0} % Removes default section numbers
\newcounter{homeworkProblemCounter} % Creates a counter to keep track of the number of problems

\newcommand{\homeworkProblemName}{}
\newenvironment{homeworkProblem}[1][Zadanie \arabic{homeworkProblemCounter}]{ % Makes a new environment called homeworkProblem which takes 1 argument (custom name) but the default is "Problem #"
\stepcounter{homeworkProblemCounter} % Increase counter for number of problems
\renewcommand{\homeworkProblemName}{#1} % Assign \homeworkProblemName the name of the problem
\section{\homeworkProblemName} % Make a section in the document with the custom problem count
\enterProblemHeader{\homeworkProblemName} % Header and footer within the environment
}{
\exitProblemHeader{\homeworkProblemName} % Header and footer after the environment
}

\newcommand{\problemAnswer}[1]{ % Defines the problem answer command with the content as the only argument
\noindent\framebox[\columnwidth][c]{\begin{minipage}{0.98\columnwidth}#1\end{minipage}} % Makes the box around the problem answer and puts the content inside
}

\newcommand{\homeworkSectionName}{}
\newenvironment{homeworkSection}[1]{ % New environment for sections within homework problems, takes 1 argument - the name of the section
\renewcommand{\homeworkSectionName}{#1} % Assign \homeworkSectionName to the name of the section from the environment argument
\subsection{\homeworkSectionName} % Make a subsection with the custom name of the subsection
\enterProblemHeader{\homeworkProblemName\ [\homeworkSectionName]} % Header and footer within the environment
}{
\enterProblemHeader{\homeworkProblemName} % Header and footer after the environment
}

\usepackage{listings} % Required for inserting code snippets
\usepackage[usenames,dvipsnames]{color} % Required for specifying custom colors and referring to colors by name

\definecolor{DarkGreen}{rgb}{0.0,0.4,0.0} % Comment color
\definecolor{highlight}{RGB}{255,251,204} % Code highlight color

% Create a command to cleanly insert a snippet with the style above anywhere in the document
\newcommand{\insertcode}[2]{\begin{itemize}\item[]\lstinputlisting[caption=#2,label=#1,style=Style1]{#1}\end{itemize}} % The first argument is the script location/filename and the second is a caption for the listing

%----------------------------------------------------------------------------------------
%	NAME AND CLASS SECTION
%----------------------------------------------------------------------------------------

\newcommand{\hmwkTitle}{Pracownia 1} % Assignment title
\newcommand{\hmwkDueDate}{} % Due date
\newcommand{\hmwkClass}{Analiza numeryczna} % Course/class
\newcommand{\hmwkClassTime}{} % Class/lecture time
\newcommand{\hmwkClassInstructor}{} % Teacher/lecturer
\newcommand{\hmwkAuthorName}{Bartosz Bednarczyk, Jan Góra - Sprawozdanie z pracowni nr 1 - Zadanie 24} % Your name

%----------------------------------------------------------------------------------------
%	TITLE PAGE
%----------------------------------------------------------------------------------------

\title{Analiza numeryczna (M) - Pracownia 1 - Zadanie P1.24\\
Analiza numeryczna iteracyjnej metody Bairstowa}
\date{Listopad 15, 2015}
\author{Bartosz Bednarczyk\\ \and Jan Góra}


%----------------------------------------------------------------------------------------

\begin{document}

\maketitle

\tableofcontents

\section{Krótki opis sprawozdania}

Najprostsze metody numeryczne bardzo często okazują się być mało wydajne, dlatego matematycy dążą do uzyskania metod o bardzo niskim czasie działania. Celem tego sprawozdania jest pokazanie jednej z nich, jaką jest iteracyjna metoda Bairstowa. Na podstawie odpowiednich testów numerycznych sprawdzone zostaną dokładność, stabilność i efektywność tej metody. Poza tym pobieżnie omówione zostaną różne warianty metody Newtona, metoda Laguerre'a oraz metoda Mullera, których wydajność porównamy z metodą Bairstowa.	
\section{Przegląd podstawowych zagadnień związanych z wielomianami}

\subsection{Podstawowe definicje}

\begin{definicja}
Wielomianem stopnia $n \in \mathbb{N}$ nad ciałem $\mathbb{K}$ będziemy nazywać przekształcenie $\mathbb{K} \mapsto \mathbb{K}$ zadane wzorem $W(x) = a_0 + a_1x + \ldots + a_n x^n$, gdzie $a_i$ to pewne  współczynniki z ciała $\mathbb{K}$ oraz $a_n \neq 0$.
\end{definicja}

\begin{definicja}
Niech $W$ będzie pewnym wielomianem (nad ciałem $\mathbb{K}$). Liczbę $a$ taką, że $W(a) = 0$, będziemy nazywać pierwiastkiem wielomianu.
\end{definicja}

\begin{remark}
Z faktu, że wielomian $W$ ma współczynniki z ciała $\mathbb{K}$, nie wynika fakt, że jego pierwiastki również będą należeć do $\mathbb{K}$. Klasycznym przykładem jest wielomian $x^2 + 1$, który ma współczynniki rzeczywiste, a jego pierwiastkami są liczby zespolone.
\end{remark}

\begin{remark}
Istnieją takie ciała $\mathbb{K}$, że dla dowolnego wielomianu stopnia większego od $0$ wszystkie jego pierwiastki należą do $\mathbb{K}$. Ciała takie będziemy nazywać algebraicznie domkniętymi. Przykładem takiego ciała jest $\mathbb{C}$, czego nie będziemy dowodzić. 
\end{remark}

Podczas całego tego sprawozdania będziemy zajmować się następującym problemem:

\begin{center}
\begin{tikzpicture}
\node [draw={black}, fill=black!10, very thick, rectangle, rounded corners, inner sep=12pt, inner ysep=12pt] (box){%
    \begin{minipage}{.9\textwidth}

	Niech $W$ będzie wielomianem. Znaleźć zbiór $ker(W) = \lbrace a \ | \ W(a) = 0 \rbrace $.
    \end{minipage}
};
\node[fill={black}, text=white, rounded corners, right=10pt] at (box.north west) {Problem znajdowania miejsc zerowych wielomianu};
\end{tikzpicture}
\end{center}

Powyższy problem, choć pozornie prosty, jest sformułowany bardzo ogólnie. Na potrzeby tej pracy od tej pory ograniczymy się tylko do ciał $\mathbb{R}$ oraz $\mathbb{C}$, choć nic nie staje na przeszkodzie by poeksperymentować z innymi ciałami. Aktualnie nie wiemy czy każdy wielomian ma pierwiastki, a jeśli ma, to jaka jest moc ich zbioru. Nie znamy również żadnych metod rozwiązywania $W(x) = 0$. By lepiej zrozumieć podane zagadnienie, przejdźmy przez ciąg różnych definicji, algorytmów, twierdzeń i lematów związanych z wielomianami (warto je zrozumieć, gdyż kolejne rozdziały będą z nich korzystać).

\begin{thm}
Każdy wielomian $W(x)$ nad $\mathbb{C}$ stopnia $n \in \mathbb{N}_+$ ma co najmniej jeden pierwiastek.
\end{thm}

\begin{proof}
To twierdzenie jest nazywane zasadniczym twierdzeniem algebry. Udowodnione w \cite{leja}, s. 105.
\end{proof}

\begin{wniosek}
$| ker(W) | \leq n$, gdzie $n$ to stopień wielomianu $W$.	
\end{wniosek}


\subsection{Postać iloczynowa wielomianu i dzielenie wielomianu}


\begin{definicja}
Wielomian $W(x)$ nazywamy podzielnym przez wielomian $P(x)$, różny od wielomianu zerowego, wtedy i tylko wtedy, gdy istnieje taki wielomian $Q(x)$, że $W(x) = Q(x) \cdot P(x)$. Wielomian $Q(x)$ nazywamy ilorazem wielomianu $W(x)$ przez $P(x)$. Mówimy, że wielomian $P(x)$ jest dzielnikiem wielomianu $W(x)$.
\end{definicja}


\begin{thm}
Dowolny wielomian $W(x)$ możemy zapisać jako $W(x) = P(x) \cdot Q(x) + R(x)$ dla pewnych wielomianów $P, Q, R$. Mówimy, że wielomian $W(x)$ jest podzielny przez $Q(x)$, jeżeli $R(x) = 0$. 
\end{thm}

\begin{thm}
Wielomian $W(x)$ jest podzielny przez wielomian $Q(x) = (x-a)$ wtedy i tylko wtedy, gdy $W(a) = 0$.	
\end{thm}

\begin{proof}
Udowodnione w \cite{kostrikin}, s. 198-199.
\end{proof}

Chcielibyśmy umieć w efektywny sposób realizować procedurę dzielenia wielomianu przez jednomiany postaci $x-a$. Służy do tego następujący algorytm:

\begin{enumerate}
\item $P(x) = a_0 + a_1 x + a_2 x^2 + \ldots + a_n x^n$.
\item Niech $\alpha = a_n$.
\item Kolejno dla $k = n-1, n-2, \ldots 0$ wykonaj $\alpha := a_k + x \alpha$.
\item Wynik to $p(x) = \alpha$.
\end{enumerate}

Dokładny opis metody oraz jej analizę możemy znaleźć w \cite{kincaid}, s. 103.\\

\subsection{Pochodna wielomianu i jej obliczanie}

\begin{definicja}
Pochodną wielomianu $p(x) = a_nx^n + a_{n-1}x^{n-1} + \ldots + a_1x + a_0$ będziemy nazywać wielomian $p'(x) = 	n \cdot a_n x^{n-1} + (n-1) a_{n-1}x^{n-2} + \ldots +  a_1$.
\end{definicja}

Wyznaczanie wielomianu w punkcie $x_0$ możemy zrealizować za pomocą schematu Hornera:

\begin{enumerate}
\item $P(x) = a_0 + a_1 x + a_2 x^2 + \ldots + a_n x^n$.
\item Niech $\alpha := a_n, \ \beta := 0$.
\item Kolejno dla $k = n-1, n-2, \ldots 0$ wykonaj $\beta := \alpha + x \beta, \ \alpha := a_k + x \alpha.$

\item Wynik to $p'(x) = \beta$.
\end{enumerate}

\subsection{Inne przydatne pojęcia matematyczne}

Oprócz wymienionych w rozdziale pojęć związanych z wielomianami, zakładać będziemy u czytelnika znajomość wielowymiarowego rachunku różniczkowego, definicji funkcji holomorficznej oraz podstawowych pojęć związanych z analizą błędów. Pojęcia te można doczytać w \cite{leja} i \cite{kincaid}. 

\section{Metoda Newtona oraz wielowymiarowa metoda Newtona}

\subsection{Opis klasycznej metody Newtona}

Klasyczną metodą Newtona zastosowaną dla pewnego punktu startowego $p$ oraz funkcji $f : \mathbb{R} \mapsto \mathbb{R}$ klasy $C^{1}$ nazywać będziemy metodę iteracyjną postaci:

$$
x_n = \left\{\begin{matrix}
p, & gdy \ n = 0\\ 
x_{n-1} - \frac{f(x_{n-1})}{f'(x_{n-1})}, & w.p.p. 
\end{matrix}\right.
$$

Można pokazać, że $x_n$ zbiega do pewnego pierwiastka funkcji $f$. Analizę klasycznej metody Newtona można znaleźć w \cite{kincaid}, s.   71-81.

\subsection{Zastosowanie klasycznej metody Newtona do szukania zer wielomianu}

Jeśli mamy wielomian o współczynnikach i pierwiastkach rzeczywistych, możemy policzyć jego pierwiastki za pomocą klasycznej metody Newtona. Podstawiamy za $f$ z poprzedniego opisu nasz wielomian, a $f'$ to jego pochodna. Po znalezieniu jednego pierwiastka (nazwijmy go $a$) dzielimy nasz wielomian przez $x-a$ i uruchamiamy program dla otrzymanego wielomianu. Proces kontynuujemy tak długo, aż dojdziemy do wielomianu o stopniu $0$.

\begin{remark}
Poniżej przedstawiamy przykładową implementację metody Newtona w języku Julia. Wartość w punkcie wielomianu i jego pochodnej możemy wyznaczyć z pomocą schematu Hornera, który był omówiony wcześniej (w kodzie przykładowym skorzystaliśmy z funkcji bibliotecznych dla większej czytelności).
\end{remark}


\lstset{language=Julia, label=DescriptiveLabel, frame=shadowbox}
\lstinputlisting{klasyczna_metoda_newtona.jl}

\begin{remark}
Powyższa metoda nie nadaje się do obliczania miejsc zerowych wielomianu, którego pierwiastki są zespolone (z powodu tego, że operujemy tutaj na tylko rzeczywistych przybliżeniach $x_n$).
\end{remark}

\subsection{Metoda Newtona dla wielu funkcji wielu zmiennych}

Załóżmy, że mamy do rozwiązania układ równań:

$$\left\{\begin{matrix}
f_1(x_1, x_2, \ldots, x_n) = 0 \\ 
f_2(x_1, x_2, \ldots, x_n) = 0 \\ 
\ldots\\ 
f_n(x_1, x_2, \ldots, x_n) = 0,
\end{matrix}\right.$$

gdzie $f_i \in \mathbb{R}^n \mapsto \mathbb{R}^n$ jest klasy $C^{1}$.

Każdą z tych funkcji możemy rozpisać ze wzoru Taylora jako:

$$0 = f_i(x_1 + h_1, x_2 + h_2, \ldots, x_n + h_n) \approx f_i(x_1, x_2, \ldots, x_n) + \sum_{j=1}^n h_j \cdot \frac{\partial f_i}{\partial x_j} (x_1, x_2, \ldots x_n)$$

Powyższy układ możemy zapisać w postaci macierzowej:

$$
\begin{pmatrix}
f_1(x_1, x_2, \ldots, x_n)\\ 
f_2(x_1, x_2, \ldots, x_n)\\ 
\ldots\\ 
f_n(x_1, x_2, \ldots, x_n)\\ 

\end{pmatrix} = \begin{pmatrix}
\frac{\partial f_1}{\partial x_1}(x_1, x_2, \ldots x_n) & \frac{\partial f_1}{\partial x_2}(x_1, x_2, \ldots x_n) &  \ldots &  & \frac{\partial f_1}{\partial x_1}(x_1, x_2, \ldots x_n) \\ 
\frac{\partial f_2}{\partial x_1}(x_1, x_2, \ldots x_n) & \frac{\partial f_2}{\partial x_2}(x_1, x_2, \ldots x_n) &  \ldots &  & \frac{\partial f_2}{\partial x_1}(x_1, x_2, \ldots x_n) \\ 
 \ldots & \ldots &  \ \ldots & \\ 
 \ldots & \ldots &  \ \ldots & \\ 
\frac{\partial f_n}{\partial x_1}(x_1, x_2, \ldots x_n) & \frac{\partial f_n}{\partial x_2}(x_1, x_2, \ldots x_n) &  \ldots &  & \frac{\partial f_n}{\partial x_1}(x_1, x_2, \ldots x_n) \\ 
\end{pmatrix} \begin{pmatrix}
h_1\\ 
h_2\\ 
\ldots \\ 
\ldots \\ 
h_n\\ 

\end{pmatrix}
$$

Aby nieco skrócić ten układ, będziemy go zapisywać jako $F(X) = -J \cdot H$.
Jeśli macierz $J$ jest nieosobliwa, to układ ma rozwiązanie w postaci:
$$ -J^{-1} \cdot F(X) = H$$

Ostatecznie wzór Newtona dla układu funkcji wielu zmiennych możemy wyrazić wzorem:

$$ X_{k+1} = X_k + H_k = X_k - J^{-1}(X_k) F(X_k)$$


\subsection{Metoda Newtona dla funkcji zespolonych}

\begin{lemat}
Dowolną funkcję analityczną $f : \mathbb{C} \mapsto \mathbb{C}$ możemy zapisać jako $$f(z) = f(x+yi) = P(x,y) + i Q(x,y),$$ gdzie $x,y \in \mathbb{R}$, $P(x,y) \in \mathbb{R}, Q(x,y) \in \mathbb{R}$	
\end{lemat}

\begin{przyklad}

$$f(z) = z^3 - 2z = f(x+iy) = (x+iy)^3 - 2(x+iy) = (x^3 - 3xy^2 - 2x) + i(3x^2y - y^3 - 2y) = P(x,y) + iQ(x,y)$$

\end{przyklad}


Niech $f(z) = P(x,y) + iQ(x,y)$. Równanie $f(z) = 0$ możemy sprowadzić do układu równań $Q(x,y) = 0$ i $P(x,y) = 0$. Taki układ równań rozwiązujemy za pomocą metody Newtona dla funkcji wielu zmiennych.

$$v_{n+1} = v_n - \frac{f(v_n)}{f'(v_n)}$$

$$
\begin{pmatrix} x_{n+1}\\ y_{n+1}\\ \end{pmatrix} =  \begin{pmatrix} x_n\\ y_n\\ \end{pmatrix} - J^{-1} \begin{pmatrix} P(x_n, y_n)\\ Q(x_n, y_n)\\ \end{pmatrix} = 
\begin{pmatrix} x_n\\ y_n\\ \end{pmatrix} - \begin{pmatrix}
\frac{\partial P}{\partial x}(x_n, y_n) & \frac{\partial P}{\partial y} (x_n, y_n)\\ 
\frac{\partial Q}{\partial x}(x_n, y_n) & \frac{\partial Q}{\partial y}(x_n, y_n)  
\end{pmatrix}^{-1} \begin{pmatrix} P(x_n, y_n)\\ Q(x_n, y_n)\\ \end{pmatrix} $$


Ponieważ wielomian jest funkcją holomorficzną, to zachodzi równanie Cauchy'ego-Riemanna:

$$\frac{\partial P}{\partial x} = \frac{\partial Q}{\partial y}, \ - \frac{\partial P}{\partial y} =\frac{\partial Q}{\partial x}$$

Oznaczając $P = P(x_n, y_n), Q = Q(x_n, y_n), P_x = \frac{\partial P}{\partial x}(x_n, y_n)$, $Q_x = \frac{\partial Q}{\partial x}(x_n, y_n)$ oraz korzystając ze wzoru na macierz odwrotną możemy uprościć wzór na metodę Newtona do postaci:

$$
\begin{pmatrix} x_{n+1}\\ y_{n+1}\\ \end{pmatrix} = \begin{pmatrix} x_n - \frac{ P P_x + Q Q_x}{Px^2 + Qx^2}\\ y_n - \frac{P P_y + Q Q_y}{Px^2 + Qx^2} \\ \end{pmatrix}
$$


\begin{remark}
Implementacja zespolonej metody Newtona jest nieco problematyczna. Musimy potrafić zaimplementować operacje na funkcjach wielu zmiennych oraz ich różniczkowanie. Przykładowy kod w języku Julia możemy otrzymać dzięki zastosowaniu biblioteki MultiPoly. Wymieniona biblioteka dostarcza nam sposobu na tworzenie nowych zmiennych wielomianu (metoda generators), obliczania wartości w danym punkcie (evaluate) oraz efektywnego różniczkowania wielomianu po zadanej zmiennej (diff).  \end{remark}


\lstset{language=Julia, label=DescriptiveLabel, frame=shadowbox}
\lstinputlisting{zespolony_newton.jl}

\section{Wybrane metody wyszukiwania miejsc zerowych wielomianu}

\subsection{Metoda Laguerre'a}

Jedną z metod iteracyjnych wyszukiwania pierwiastków wielomianu używanych w nowoczesnych systemach informatycznych jest metoda Laguerre'a. 

Niech $p(z)$ będzie wielomianem stopnia $n$, którego pierwiastki mamy znaleźć. Kolejne kroki w metodzie wykonujemy za pomocą nastepujących wzorów:

$$ A = \frac{-p'(z)}{p(z)}, \ B = A^2 - \frac{p''(z)}{p(z)}, \ C = \frac{A \pm \sqrt{(n-1)(nB - A^2}}{n}, \ z_{nowe} = z + \frac{1}{C}$$ 

Metoda Laguerre'a jest bardzo efektywnym algorytmem, ponieważ w okolicach pojedynczego pierwiastka wielomianu $p$ jest zbieżna sześcienne. Dokładną analizę tej metody pozostawiamy czytelnikowi do przeczytania w \cite{kincaid}, s. 112-116.

\subsection{Metoda Mullera}

Metoda Mullera jest modyfikacją metody stycznych. Zamiast przybliżać nasz wielomian $f$ funkcją liniową, będziemy go aproksymować funkcją kwadratową.

Rozważmy trzy punkty $x_0, x_1, x_2$ wraz z wartościami funkcji $f$ w tych punktach. Przyjmujmy, że $x_2$ jest aktualnym przybliżeniem rozwiązania. Oznaczmy $z = x-x_2, \ h_0 = x_0 - x_2, \ h_1 = x_1-x_2$.

Oznaczmy szukaną parabolę przez $g(z) = az^2 + bz + c$. Z definicji paraboli w punkcie $z - x_k$ dostajemy, że 

$$2a = f''(x_k), \ b = f'(x_k), \ c = f(x_k),$$

co prowadzi do wzoru

$$x_{k+1} = x_k - \frac{2f(x_k)}{f'(x_k) + sgn(f'(x_k)) \cdot \sqrt{(f'(x_k))^2 - 2f(x_k)f''(x_k)}}$$

Więcej na temat tej metody oraz jej modyfikacji można poczytać w skrypcie \cite{tajewski}.


\section{Metoda Bairstowa}

\subsection{Opis metody Bairstowa}

Ostatnią i zarazem najważniejsza metodą, którą omówimy w sprawozdaniu, będzie metoda Bairstowa. Wiemy, że nawet jeśli wielomian ma współczynniki rzeczywiste, to może mieć pierwiastki zespolone (np. $x^2 + 1$). Metoda Bairstowa pozwala na obliczenie wszystkich pierwiastków bez użycia arytmetyki zespolonej.


\begin{lemat}
Jeżeli $w$ jest pierwiastkiem nierzeczywistym wielomianu $p(z)$, a $p(z)$ jest wielomianem o współczynnikach rzeczywistych, to pierwiastkiem $p(z)$ jest również $\overline{w}$. Iloczyn $(x-w)(x-\overline{w})$ jest czynnikiem kwadratowym o współczynnikach rzeczywistych.
\end{lemat}

\begin{proof}
Udowodnione w \cite{kincaid}, s. 108.	
\end{proof}

Zauważmy, że pierwiastki zespolone możemy wyszukiwać parami. Zamiast wyszukiwać pierwiastki pojedynczo, będziemy wyszukiwać dwumianu postaci $z^2 - uz - v$.

\begin{lemat}
Dowolny wielomian $p(z) = a_nx^n  + a_{n-1}x^{n-1} + \ldots a_0$ możemy zapisać w postaci: 

$$p(z) = \left(b_nx^{n-2} + b_{n-1}z^{n-3} + \ldots + b_3 z + b_2 \right) \left( z^2 - uz - v \right) + b_1(z-u) + b_0 $$	

Wielomian $ \left(b_nx^{n-2} + b_{n-1}z^{n-3} + \ldots + b_3 z + b_2 \right) $ będziemy dalej oznaczać jako $Q(z, u, v)$.
Powyższe współczynniki możemy obliczać rekurencyjnie według wzorów:
$$b_{n+1} = b_{n+2} = 0, \ \ b_k = ub_{k+1} + vb_{k+2} \ (n \geq k \geq 0).$$

\end{lemat}

\begin{proof}
Dowód w \cite{kincaid}, s. 109.	
\end{proof}

Chcemy, by nasz wyjściowy wielomian był podzielny przez $z^2 - uz - v$. Zatem musi zachodzić $b_0 = b_1 = 0$. Potraktujmy podane współczynniki jako funkcje zmiennych $u,v$. Wtedy dostajemy do rozwiązania układ równań:
$$
\left\{\begin{matrix}
b_0(u + h_1, v + h_1) = 0 & \\ 
b_1(u + h_2, v + h_2) = 0 & 
\end{matrix}\right.
$$

Podany układ możemy rozwiązać przedstawioną wcześniej metodą Newtona dla wielu funkcji wielu zmiennych. Po znalezieniu współczynników $u, v$ dzielimy wyjściowy wielomian przez otrzymany dwumian i kontynuujemy proces wyszukiwania pierwiastków dla mniejszego wielomianu (z uwzględnieniem tego, że przypadki dla wielomianu stopnia $0$ i $1$ traktujemy osobno).

\subsection{Analiza teoretyczna metody Bairstowa}

\begin{lemat}
Metoda Bairstowa jest zbieżna lokalnie.	
\end{lemat}

\begin{proof}
Wynika to bezpośrednio z tego, że metoda Newtona jest zbieżna lokalnie.	
\end{proof}

Głównym założeniem w lokalnej zbieżności metody Bairstowa jest to, że jakobian wyliczany przy metodzie Newtona się nie zeruje dla podanych wcześniej punktów startowych i kolejnych przybliżeń. Zastanówmy się w jaki sposób zerowanie się jakobianu zależy od punktów startowych oraz pierwiastków wielomianu.

\begin{thm}
Niech $u, v$ będą dowolnie wybranymi liczbami rzeczywistymi. Jakobian dla algorytmu Bairstowa jest macierzą odwracalną wtedy i tylko wtedy, gdy $z^2 - uz - v$ oraz wielomian $Q(z,u,v)$ nie mają wspólnych pierwiastków. Rząd jakobianu jest jeden wtedy i tylko wtedy, kiedy liczba wspólnych pierwiastków (z krotnościami) jest równa jeden. Jakobian się zeruje wtedy i tylko wtedy, gdy $z^2 - uz - z$ dzieli $Q(z,u,v)$.
\end{thm}

\begin{thm}
Załóżmy, że $P(z) = Q(z, u, v) (x^2 - u^{\star} z - v^{\star})$ i załóżmy, że wyrażenia po prawej stronie nie mają wspólnego pierwiastka. Wtedy istnieje dodatnia liczba $d$ taka, że ciąg $(u_k, v_k)$ generowany przez metodę Bairstowa jest zbieżny kwadratowo do $(u^{\star}, v^{\star})$, gdzie $|u_0 - u^{\star}| < d \wedge |v_0 - v^{\star}| < d$. 
\end{thm}

\begin{proof}
Twierdzenia te zostały udowodnione przez autorów Tibora Fialę oraz Annę Krebsz w 1987 roku. Kompletne dowody można przeczytać w \cite{krebsz}.
\end{proof}

Analiza zbieżności oraz rozbieżności metody Bairstowa stanowiła podstawę do napisaniu kilku (choć niestety niewielu) prac naukowych. Zainteresowanego czytelnika odsyłamy do \cite{krebsz}, \cite{Gabler} oraz \cite{Glasson}. 

\subsection{Przykład rozbieżności metody Bairstowa}

Rozważmy wielomian postaci $P(x) = (x^2 + ux + v)(x^2 + ux + w) + (w-v)^2.$
Jakobian dla metody Bairstowa w punkcie $(u,v)$ będzie wyglądał następująco:
$$
J(u,v) = \begin{pmatrix}
v-w & 0\\ 
0 & v-w 
\end{pmatrix}
$$

Jeśli uruchomimy metodę Bairstowa dla np. $u = 3, v = 1, w = 2$ to dostaniemy wielomian $x^4 + 6x^3 + 12x^2 + 9x +3,$ dla którego ciąg przybliżeń pierwiastka będzie cykliczny.

\begin{table}[h!]
\centering
\caption{Metoda Bairstowa dla powyższej funkcji}
\label{my-label}
\begin{tabular}{|l|l|l|l|l|}
\hline
Iteracja & $x_0$           & $x_1$               & $x_2$           & $x_3$              \\ \hline
1        & -2.6180339884   & -3.8196601113e-01   & -2.6180339884   & -3.8196601113e-01   \\ \hline
2        & -2.0            & -1.0                & -2.0            & -1.0                \\ \hline
3        & -2.6180339884 & -3.8196601113e-01 & -2.6180339884 & -3.8196601113e-01 \\ \hline
4       & -2.0            & -1.0                & -2.0            & -1.0                \\ \hline

\end{tabular}
\end{table}

\subsection{Implementacja metody Bairstowa}

Jeżeli oznaczymy sobie $c_k := \frac{\partial b_k}{\partial u}, \ d_k := \frac{\partial b_{k-1}}{\partial v}$ to dostajemy związki:

$c_k = d_{k+1} + uc_{k+1} + vc_{j+2} \ (c_{n+1} = c_n = 0)$ oraz $d_k = b_{k+1} + u d_{k+1} + v d_{k+2} (d_{n+1} = d_n = 0)$.

Rozwiązujemy układ równań

$$
\left\{\begin{matrix}
b_0(u, v) + \frac{\partial b_0}{\partial u} \delta u + \frac{\partial b_0}{\partial v} \delta v  = 0 & \\ 
b_1(u, v) + \frac{\partial b_1}{\partial u} \delta u + \frac{\partial b_1}{\partial v} \delta v  = 0 & \\ \end{matrix}\right.
$$

Rozwiązaniem powyższego układu jest 
$$\delta u = (c_1b_1 - c_2b_0)/J, \ \delta v = (c_1 b_0 - c_0 b_1)/J, \ J = c_0c_2 - c_1^2$$

Metodę Bairstowa możemy zapisać w postaci listy kroków:

\begin{enumerate}
\item $b_n := a_n$
\item $c_n := 0$
\item $c_{n-1} := a_n$
\item for $j = 1$ to $M$ do
	\begin{enumerate}
	\item $b_{n-1} := a_{n-1} + ub_n$
	\item for $k = n-2$ to $0$ step $-1$ do
		\begin{enumerate}
		\item $b_k := a_k + ub_{k+1} + vb_{k+2}$
		\item $c_k := b_{k+1} + uc_{k+1} + vc_{k+2}$
		\end{enumerate}
	\item end do

\item $J := c_0c_2 - c_1^2$
\item $u := u + (c_1b_1 - c_2b_0)/J$
\item $v := v + (c_1b_0 - c_0b_1)/J$
\item output $j, u, v, b_0, b_1$
\end{enumerate}
\item end do	
\end{enumerate}

Pseudokod zapożyczony z opisu metody Bairstowa z \cite{kincaid}.

\section{Testy numeryczne}


\begin{thebibliography}{99}
\bibitem{leja} Leja Franciszek,
\emph{Funkcje zespolone},
Warszawa, PWN, 1976.

\bibitem{kostrikin} Aleksiej I. Kostrikin, przekł. Jerzy Trzeciak,
\emph{Wstęp do algebry. Podstawy algebry},
Warszawa, PWN, 2008.

\bibitem{kincaid} David Kincaid, Ward Cheney, przekł. Stefan Paszkowski,
\emph{Analiza numeryczna},
Warszawa, WNT, 2006.

\bibitem{complexnewton} 
Lily Yau, Adi Ben-Israel,
\emph{The Newton and Halley Methods for Complex Roots},
The American Mathematical Monthly 105, 1998, s. 806–818.

\bibitem{krebsz} 
Tibor Fiala, Anna Krebsz,
\emph{On the Convergence and Divergence of Bairstow's Method},
Journal Numerische Mathematik, Volume 50 Issue 4, 1987, s. 477-482.

\bibitem{Gabler} 
Wolfgang Gabler
\emph{Invariances and convergence properties of Bairstow's method},
International Journal of Pure and Applied Mathematics Volume 27 No. 4, 2006, s. 523-576.

\bibitem{Glasson} 
Sofo, Anthony and Glasson, Alan,
\emph{Singularities in Bairstow’s method},
Gazette of the Australian Mathematical Society, 37 (2), s. 93-100.
\bibitem{tajewski}
Piotr Tatjewski,
\emph{Równania nieliniowe i zera wielomianów,}
Skrypt do wykładu Metody Numeryczne.

\end{thebibliography}

\end{document}