\documentclass{article}

\usepackage{polski}
\usepackage[utf8]{inputenc}

\usepackage{fancyhdr} % Required for custom headers
\usepackage{lastpage} % Required to determine the last page for the footer
\usepackage{extramarks} % Required for headers and footers
\usepackage[usenames,dvipsnames]{color} % Required for custom colors
\usepackage{graphicx} % Required to insert images
\usepackage{listings} % Required for insertion of code
\usepackage{courier} % Required for the courier font
\usepackage{lipsum} % Used for inserting dummy 'Lorem ipsum' text into the template\
\usepackage{amsfonts}
\usepackage{amsthm}
\usepackage{hyperref}
\usepackage{tikz}
\usepackage{amsmath}
\usepackage{pdfpages}
\usepackage{mathtools}


\newtheorem{thm}{Twierdzenie}
\newtheorem{remark}{Uwaga}
\newtheorem{lemat}{Lemat}


\newenvironment{prooff}{\paragraph{Dowód:}}{\hfill$\square$}
\newenvironment{rozw}{\paragraph{Rozwiązanie:}}{\hfill}

% Margins
\topmargin=-0.45in
\evensidemargin=0in
\oddsidemargin=0in
\textwidth=6.5in
\textheight=9.0in
\headsep=0.25in

\linespread{1.1} % Line spacing

% Set up the header and footer
\pagestyle{fancy}
\lhead{\hmwkAuthorName} % Top left header
\chead{\hmwkClass\ (\hmwkClassInstructor\ \hmwkClassTime): \hmwkTitle} % Top center head
\rhead{\firstxmark} % Top right header
\lfoot{\lastxmark} % Bottom left footer
\cfoot{} % Bottom center footer
\rfoot{Page\ \thepage\ of\ \protect\pageref{LastPage}} % Bottom right footer
\renewcommand\headrulewidth{0.4pt} % Size of the header rule
\renewcommand\footrulewidth{0.4pt} % Size of the footer rule

\setlength\parindent{0pt} % Removes all indentation from paragraphs
%----------------------------------------------------------------------------------------
%	DOCUMENT STRUCTURE COMMANDS
%	Skip this unless you know what you're doing
%----------------------------------------------------------------------------------------

% Header and footer for when a page split occurs within a problem environment
\newcommand{\enterProblemHeader}[1]{
\nobreak\extramarks{#1}{#1 continued on next page\ldots}\nobreak
\nobreak\extramarks{#1 (continued)}{#1 continued on next page\ldots}\nobreak
}

% Header and footer for when a page split occurs between problem environments
\newcommand{\exitProblemHeader}[1]{
\nobreak\extramarks{#1 (continued)}{#1 continued on next page\ldots}\nobreak
\nobreak\extramarks{#1}{}\nobreak
}

\setcounter{secnumdepth}{0} % Removes default section numbers
\newcounter{homeworkProblemCounter} % Creates a counter to keep track of the number of problems

\newcommand{\homeworkProblemName}{}
\newenvironment{homeworkProblem}[1][Zadanie \arabic{homeworkProblemCounter}]{ % Makes a new environment called homeworkProblem which takes 1 argument (custom name) but the default is "Problem #"
\stepcounter{homeworkProblemCounter} % Increase counter for number of problems
\renewcommand{\homeworkProblemName}{#1} % Assign \homeworkProblemName the name of the problem
\section{\homeworkProblemName} % Make a section in the document with the custom problem count
\enterProblemHeader{\homeworkProblemName} % Header and footer within the environment
}{
\exitProblemHeader{\homeworkProblemName} % Header and footer after the environment
}

\newcommand{\problemAnswer}[1]{ % Defines the problem answer command with the content as the only argument
\noindent\framebox[\columnwidth][c]{\begin{minipage}{0.98\columnwidth}#1\end{minipage}} % Makes the box around the problem answer and puts the content inside
}

\newcommand{\homeworkSectionName}{}
\newenvironment{homeworkSection}[1]{ % New environment for sections within homework problems, takes 1 argument - the name of the section
\renewcommand{\homeworkSectionName}{#1} % Assign \homeworkSectionName to the name of the section from the environment argument
\subsection{\homeworkSectionName} % Make a subsection with the custom name of the subsection
\enterProblemHeader{\homeworkProblemName\ [\homeworkSectionName]} % Header and footer within the environment
}{
\enterProblemHeader{\homeworkProblemName} % Header and footer after the environment
}

\usepackage{listings} % Required for inserting code snippets
\usepackage[usenames,dvipsnames]{color} % Required for specifying custom colors and referring to colors by name

\definecolor{DarkGreen}{rgb}{0.0,0.4,0.0} % Comment color
\definecolor{highlight}{RGB}{255,251,204} % Code highlight color

\lstdefinestyle{Style1}{ % Define a style for your code snippet, multiple definitions can be made if, for example, you wish to insert multiple code snippets using different programming languages into one document
language=Perl, % Detects keywords, comments, strings, functions, etc for the language specified
backgroundcolor=\color{highlight}, % Set the background color for the snippet - useful for highlighting
basicstyle=\footnotesize\ttfamily, % The default font size and style of the code
breakatwhitespace=false, % If true, only allows line breaks at white space
breaklines=true, % Automatic line breaking (prevents code from protruding outside the box)
captionpos=b, % Sets the caption position: b for bottom; t for top
commentstyle=\usefont{T1}{pcr}{m}{sl}\color{DarkGreen}, % Style of comments within the code - dark green courier font
deletekeywords={}, % If you want to delete any keywords from the current language separate them by commas
%escapeinside={\%}, % This allows you to escape to LaTeX using the character in the bracket
firstnumber=1, % Line numbers begin at line 1
frame=single, % Frame around the code box, value can be: none, leftline, topline, bottomline, lines, single, shadowbox
frameround=tttt, % Rounds the corners of the frame for the top left, top right, bottom left and bottom right positions
keywordstyle=\color{Blue}\bf, % Functions are bold and blue
morekeywords={}, % Add any functions no included by default here separated by commas
numbers=left, % Location of line numbers, can take the values of: none, left, right
numbersep=10pt, % Distance of line numbers from the code box
numberstyle=\tiny\color{Gray}, % Style used for line numbers
rulecolor=\color{black}, % Frame border color
showstringspaces=false, % Don't put marks in string spaces
showtabs=false, % Display tabs in the code as lines
stepnumber=5, % The step distance between line numbers, i.e. how often will lines be numbered
stringstyle=\color{Purple}, % Strings are purple
tabsize=2, % Number of spaces per tab in the code
}

% Create a command to cleanly insert a snippet with the style above anywhere in the document
\newcommand{\insertcode}[2]{\begin{itemize}\item[]\lstinputlisting[caption=#2,label=#1,style=Style1]{#1}\end{itemize}} % The first argument is the script location/filename and the second is a caption for the listing

%----------------------------------------------------------------------------------------
%	NAME AND CLASS SECTION
%----------------------------------------------------------------------------------------

\newcommand{\hmwkTitle}{Lista 2} % Assignment title
\newcommand{\hmwkDueDate}{} % Due date
\newcommand{\hmwkClass}{Matematyka dyskretna} % Course/class
\newcommand{\hmwkClassTime}{Czw 16-19} % Class/lecture time
\newcommand{\hmwkClassInstructor}{Krzysztof Nowicki} % Teacher/lecturer
\newcommand{\hmwkAuthorName}{Bartosz Bednarczyk} % Your name

%----------------------------------------------------------------------------------------
%	TITLE PAGE
%----------------------------------------------------------------------------------------

\title{
\vspace{2in}
\textmd{\textbf{\hmwkClass:\ \hmwkTitle}}\\
\normalsize\vspace{0.1in}\small{Due\ on\ \hmwkDueDate}\\
\vspace{0.1in}\large{\textit{\hmwkClassInstructor\ \hmwkClassTime}}
\vspace{3in}
}

\author{\textbf{\hmwkAuthorName}}
\date{} % Insert date here if you want it to appear below your name

%----------------------------------------------------------------------------------------

\begin{document}


%----------------------------------------------------------------------------------------
%	TABLE OF CONTENTS
%----------------------------------------------------------------------------------------

\begin{center}
\begin{tikzpicture}
\node [draw={black}, fill=black!10, very thick, rectangle, rounded corners, inner sep=12pt, inner ysep=12pt] (box){%
    \begin{minipage}{.9\textwidth}
    Rozwiąż zależności rekurencyjne

    \begin{itemize}
    \item $a_0 = 1, \ a_{n+1} = (n+1)a_n + 1$,
    \item $b_0 = \frac{1}{2}, \ b_n = \frac{(n-2)b_{n-1} + 1}{n}$,
    \item $c_0 = 0, \ c_n = \frac{(n+2)c_{n-1} + n + 2}{n}$,
    \item $d_0 = 1, \ d_1 = 2, \ d_n =  \frac{(n-2)! d_{n-1}d_{n-2}}{n}$
    \end{itemize}

    \end{minipage}
};
\node[fill={black}, text=white, rounded corners, right=10pt] at (box.north west) {Zadanie 7};
\end{tikzpicture}
\end{center}

%----------------------------------------------------------------------------------------

\begin{rozw}

Zgadnijmy rozwiązania powyższych zależności i uzasadnijmy je indukcyjnie.

\begin{itemize}

\item $a_n = 1 + \sum_{i=0}^{n-1}{\frac{n!}{i!}}$
\begin{proof}

Niech $X \lbrace n \in \mathbb{N} \ | \ a_n = 1 + \sum_{i=0}^{n-1}{\frac{n!}{i!}}$. Zauważmy, że $0 \in X$. Weźmy dowolne $n \in \mathbb{N}$ i załóżmy, że $n \in X$. Pokażmy, że $n+1 \in X$.

$$a_{n+1} = (n+1)a_n + 1 = (n+1) \left( 1 + \sum_{i=0}^{n-1}{\frac{n!}{i!}} \right) + 1 = 1 + \left( (n+1) + (n+1)\sum_{i=0}^{n-1}{\frac{n!}{i!}}  \right) =
$$

$$
= 1 + \left( (n+1) + \sum_{i=0}^{n-1}{\frac{(n+1)!}{i!}}  \right) = 1 + \sum_{i=0}^{n}{\frac{(n+1)!}{i!}} 
$$

Zatem $n+1 \in X$. Na mocy zasady indukcji dostajemy $X = \mathbb{N}$. Zatem podany wzór jest prawdziwy dla dowolnego naturalnego $n$.

\end{proof}

\item $b_n = \frac{1}{2}$
\begin{proof}

Niech $X = \lbrace n \in \mathbb{N} \ | \ b_n = \frac{1}{2} \rbrace$. Oczywiście $0 \in X$. Weźmy dowolne $n \in \mathbb{N}$. Załóżmy, że $n \in X$ i pokażmy, że $n+1 \in X$.

$$b_{n+1} =  \frac{(n-1)b_{n} + 1}{n+1} = \frac{ \frac{1}{2} \cdot (n-1) + 1 }{n+1}  = \frac{1}{2}$$

$n+1 \in X$, a zatem na mocy zasady indukcji matematycznej podany wzór jest prawdziwy dla dowolnego $n \in \mathbb{N}$.

\end{proof}

\item $c_n = n^2 + 2n$

\begin{proof}

Niech $X = \lbrace n \in \mathbb{N} \ | \ c_n = n^2 + 2n \rbrace$. Oczywiście $0 \in X$. Weźmy dowolne $n \in \mathbb{N}$. Załóżmy, że $n \in X$ i pokażmy, że $n+1 \in X$.

$$c_{n+1} = \frac{(n+1+2)c_{n} + n+1 + 2}{n+1} = \frac{(n+3) (n^2+2n) + n+3}{n+1} =$$

$$= \frac{ (n+3)(1+ n^2 + 2n)}{n+1} = (n+1)(n+3) = (n+1)^2 + 2(n+1)$$

Z powyższych rachunków wynika, że $n+1 \in X$, a zatem na mocy zasady indukcji matematycznej podany wzór jest prawdziwy dla dowolnego $n \in \mathbb{N}$.

\end{proof}

\item $d_n = \frac{2^{F_n}}{n!}$
\begin{proof}

Niech $X = \lbrace n \in \mathbb{N} \ | \ d_n = \frac{2^{F_n}}{n!} \rbrace$. Zauważmy, że $0,1 \in X$. Weźmy dowolne $2 < n \in \mathbb{N}$. Załóżmy, że wszystkie $k < n$ należą do zbioru $X$. Obliczmy $d_{n}$.

$$d_n = \frac{(n-2)! d_{n-1}d_{n-2}}{n} = \frac{(n-2)! \cdot \frac{2^{F_{n-1}}}{(n-1)!} \cdot \frac{2^{F_{n-2}}}{(n-2)!} } {n} = \frac{ 2^{F_{n-1}} \cdot 2^{F_{n-2}} }{n!} = \frac{2^{F_n}}{n!}$$

Zatem $n \in X$. Na mocy zasady indukcji $X = \mathbb{N}$, czyli dla każdego naturalnego $n$ prawdą jest $d_n = \frac{2^{F_n}}{n!}$.

\end{proof}

\end{itemize}


\end{rozw}


\end{document}